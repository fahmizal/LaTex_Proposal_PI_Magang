\chapter[LINGKUP PERLAKSANAAN]{\\ LINGKUP PERLAKSANAAN}

\section{Waktu dan Tempat Praktik Industri}

Masa kegiatan praktik industri akan dilaksanakan berdasarkan dengan jadwal pada Program Studi Sarjana Terapan {\prodi}, {\departemen}, {\fakultas}, {\universitas} yaitu pada:

\begin{tabbing}
    \hspace{2em} \= {- Waktu} \hspace{1em} \= : \hspace{1em} \= {\waktuPelaksanaan} \\
    \> {- Tempat} \> : \> {\perusahaan} \\ 
    \> {- Almaat} \> : \> {\alamatPerusahaan} \\
\end{tabbing}

Apabila terdapat perubahan jadwal kami akan segera memberikan pemberitahuan kepada perusahaan Bapak/Ibu pimpinan.

\section{Peserta Praktik Industri}

Peserta praktik industri adalah Mahasiswa Program Studi Sarjana Terapan {\prodi}, {\departemen}, {\fakultas}, {\universitas} yang berjumlah dua orang (curriculum vitae terlampir), yaitu:

\begin{enumerate}
    \item {\penulisPertama} ({\nimPertama})
    \item {\penulisKedua} ({\nimKedua})
\end{enumerate}

\section{Pembimbing Praktik Industri}

Pada pelaksanaan Praktik Industri, mahasiswa dibimbing oleh:

\begin{enumerate}[label=\Alph*]
    \item Dosen pembimbing yang diwakili oleh pihak dosen Program Studi Sarjana Terapan {\prodi}, {\departemen}, {\fakultas}, {\universitas}. Selama praktik industri mahassiwa selalu berhubungan dengan dosen pembimbing mulai dari persiapan, konsultasi dan hingga evaluasi akhir.
    \item Pembimbing dari pihak {\perusahaan} selama praktik industri, mahasiswa dapat berhubungan dengan teknisi/operator/sarjana lapangan yang memberikan informasi, ilmu, dan pengalaman dibidang industri serta memberi evaluasi Akhir terhadap kinerja mahasiswa selama pelaksanaan kegiatan.
\end{enumerate}

\section{Ruang Lingkup Praktik Industri}

Ruang Lingkup pada praktik industri yaitu:

\subsection{Orientasi}

Kegiatan pertama dalam memulai praktik industri yaitu orientasi terhadap perusahaan atau Industri, orientasi atau pengenalan ini menyangkut sejarah berdiri, filosofi perusahaan, fokus utama bidang terkait, sistem pekerjaan, dan instrumen-instrumen yang digunakan dalam perusahaan maupun industri tersebut.

\subsection{Pengenalan Lapangan}

Dalam pengenalan lapangan ini meliputi pembelajaran terkait pada bidang elektro, sistem pengukuran yang digunakan oleh perusahaan atau industri, serta sistem K3 pada proses produksi hingga distribusi produk {\perusahaan}. Topik pembahasan praktik industri disesuaikan dengan persetujuan dan kesepakatan {\perusahaan}. Namun, dalam kesempatan praktik industri ini diajukan topik pembahasan praktik industri terkait bidang elektro.

\subsection{Tugas Khusus}

Diberikan dalam kegiatan atau proyek yang dapat menambah pengetahuan mahasiswa yang disesuaikan dengan kebutuhan {\perusahaan}.

\section{Metode Pengambilan Data}

\subsection{Data Primer}

Data primer merujuk pada informasi yang dikumpulkan secara langsung dari pengamatan atau pengujian di lapangan, atau dengan melaksanakan sebagian dari pekerjaan itu sendiri sebagai pembanding. Data primer diperoleh melalui proses sebagai berikut:

\subsubsection{Metode Survei}

Metode ini dilakukan dengan cara berinteraksi langsung, baik dengan mengajukan pertanyaan kepada pembimbing, petugas yang bertanggung jawab di bagian terkait, atau operator yang sedang bertugas, untuk mendapatkan pemahaman yang lebih dalam dan pengalaman praktis terkait dengan materi yang dipelajari.

\subsubsection{Metode Observasi}

Metode ini dilaksanakan dengan melakukan tindakan, mengamati dengan cermat, dan mencatat secara teratur mengenai fenomena yang sedang dihadapi.


\subsection{Data Sekunder}

Data Sekunder diperoleh dengan cara sebagai berikut:

\subsubsection{Data Internal}

Data diperoleh berdasarkan buku atau laporan yang tersedia di {\perusahaan}.

\subsubsection{Data Eksternal}

Data yang diperoleh berasal dari literatur-litertur yang tidak terkait dengan {\perusahaan}.


\section{Rencana Kegiatan Praktik Industri}

\begin{tabular}{|c|m{8.5cm}|m{4cm}|}
    \hline
    \textbf{No.} & \textbf{Kegiatan} & \textbf{Waktu} \\ \hline
    1 & Pengenalan Lingkungan Kerja, terkait:
    \begin{itemize}
        \item Informasi Umum Perusahaan
        \item Pendeskipsian Unit dan Divisi Kerja
        \item Peraturan Perusahaan (\textit{Company Rules})
        \item Etika Mahasiswa Praktik Industri (KP/OJT)
        \item Lain – lain.
    \end{itemize}
    & \multirow{4}{4cm}{\textit{Waktu disesuaikan dengan kebijakan perusahaan atau industri}} \\ \cline{1-2}
    2 & Fokus pada unit kerja sesuai dengan pengetahuan dan keahlian yang ingin diperluas dalam bidang elektro, dengan melibatkan interaksi antara mahasiswa dan mentor serta personel lainnya. & \\ \cline{1-2}
    3 & Pelatihan Praktik Industri (PI/OJT) dan observasi secara langsung di lapangan & \\ \cline{1-2}
    4 & Penyusunan dan presentasi laporan PI/OJT pada perusahaan yang bersangkutan. & \\ \hline
\end{tabular}

\section{Akomodasi dan Perlengkapan Praktik Industri}

Kebijakan terkait akomodasi, tunjangan, pemberangkatan, dan kedatangan mahasiswa, serta kebutuhan mereka selama Praktik Industri, diatur sesuai dengan kebijakan yang ditetapkan oleh {\perusahaan}.

\section{Laporan}

Seluruh aktivitas yang diamati dan dilakukan oleh mahasiswa selama praktik industri akan direkam dalam laporan tertulis yang disusun secara terstruktur dan teratur, mengikuti pedoman serta standar etika penulisan ilmiah.