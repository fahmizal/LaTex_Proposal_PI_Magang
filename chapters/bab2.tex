\chapter[METODOLOGI]{\\ METODOLOGI}

\section{Objek Pengamatan}

Objek pengamatan untuk penelitian praktik industri pada {\perusahaan}, yang berfokus pada sistem kontrol dan instrumentasi industri. Kami akan mempelajari cara kerja suatu proses dalam mengendalikan instrumen-instrumen yang terhubung secara otomatis maupun semi-otomatis. Selain itu, kami juga akan meneliti langkah-langkah untuk melakukan pengamanan dan menangani kesalahan dalam cara kerja dan interaksi instrumen-instrumen di {\perusahaan}.

\section{Pengertian Sistem Kontrol Proses dan Instrumentasi Industri}

Dalam industri, terdapat kebutuhan untuk mengelola variabel-variabel yang memiliki persyaratan khusus, seperti tingkat ketelitian yang tinggi, harga yang stabil dalam periode waktu tertentu, variasi nilai dalam rangkaian tertentu, hubungan tetap antara dua variabel, atau suatu variabel sebagai fungsi dari variabel lainnya. Hanya melakukan pengukuran tidaklah cukup; pengendalian juga diperlukan agar persyaratan-persyaratan tersebut dapat terpenuhi. Oleh karena itu, diperkenalkan konsep pengendalian yang disebut sistem kontrol proses. Untuk memahami sistem kontrol proses secara menyeluruh, penting untuk memahami beberapa definisi, yaitu sistem, kontrol, proses, dan sistem kontrol proses. Definisi-definisi tersebut dapat diuraikan sebagai berikut:

\begin{enumerate}[label=\alph*]
    \item Sistem: Sebuah kumpulan elemen yang terkait dan berinteraksi secara bersama-sama dalam satu kesatuan untuk mengeksekusi suatu proses dengan tujuan mencapai hasil utama.

    \item Kontrol: Suatu kegiatan yang bersifat mengontrol, mengatur, dan mengawasi suatu objek atau proses untuk memastikan kinerja yang diinginkan dan menjaga situasi agar tetap terkendali.

    \item Proses: Perubahan yang berurutan dan berlangsung secara kontinyu menuju keadaan akhir tertentu.

    \item Sistem Kontrol Proses: Rangkaian aksi yang saling berkaitan dan memiliki fungsi dalam melakukan transformasi materi adalah suatu serangkaian langkah atau tindakan yang terhubung satu sama lain dan bertujuan untuk mengubah atau mengolah materi dari satu bentuk ke bentuk lainnya.

    \item Instrumentasi Industri: Proses pengaturan atau pengendalian dilakukan terhadap satu atau beberapa variabel sehingga nilainya berada dalam rentang atau nilai tertentu. Contoh variabel fisik termasuk tekanan, aliran, suhu, ketinggian, pH, kepadatan, kecepatan, dan lain-lain.
    
\end{enumerate}

Instrumentasi dan kontrol adalah penggunaan teknik instrumen atau peralatan untuk mengontrol dan memantau sifat fisik dan kimia suatu material atau lingkungan, dengan tujuan meningkatkan efisiensi dan nilai tambah. Dalam konteks {\perusahaan}, penerapan instrumentasi dan kontrol memiliki peran penting dalam memastikan kenyamanan, efisiensi energi, dan keamanan pada bangunan perumahan dan permukiman. 

{\perusahaan} fokus pada pengembangan teknologi yang mendukung perumahan dan permukiman, salah satunya adalah penerapan sistem smart home. Sistem ini memanfaatkan sensor dan perangkat kontrol untuk mengelola berbagai aspek rumah, seperti pencahayaan, suhu, keamanan, dan konsumsi energi secara otomatis. Melalui integrasi dengan teknologi monitoring berbasis website atau aplikasi, penghuni rumah dapat memantau dan mengontrol rumah mereka dari jarak jauh, memberikan kenyamanan dan efisiensi yang lebih tinggi. Sebagai contoh, dalam sistem smart home yang kami rancang, sensor suhu dan kelembaban digunakan untuk mengontrol HVAC (Heating, Ventilation, and Air Conditioning) secara otomatis. Sensor gerak dan pintu digunakan untuk sistem keamanan rumah, yang dapat memberikan notifikasi real-time kepada penghuni melalui aplikasi mobile. Selain itu, sistem pencahayaan pintar dapat diatur untuk menyesuaikan intensitas cahaya berdasarkan waktu dan aktivitas di rumah, yang tidak hanya meningkatkan kenyamanan tetapi juga menghemat energi. 

Pengalaman kami dalam merancang sistem smart home ini meliputi pengembangan perangkat lunak untuk monitoring dan kontrol yang terintegrasi dengan website dan aplikasi mobile. Sistem ini memungkinkan pengguna untuk mendapatkan informasi real-time mengenai kondisi rumah mereka dan melakukan penyesuaian yang diperlukan dari mana saja dan kapan saja. Teknologi ini tidak hanya meningkatkan kualitas hidup penghuni, tetapi juga mendukung upaya konservasi energi dan keamanan lingkungan perumahan. Oleh karena itu, kami berencana untuk mengikuti praktik industri di {\perusahaan} guna memahami lebih lanjut tentang aplikasi instrumentasi dan kontrol dalam konteks perumahan dan permukiman. Kami berharap kegiatan ini dapat memberikan pengalaman berharga dan relevan bagi kami dalam dunia kerja, serta berkontribusi pada pengembangan teknologi perumahan yang lebih cerdas dan efisien.
