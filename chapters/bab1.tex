\chapter[PENDAHULUAN]{\\ PENDAHULUAN}

\section{Latar Belakang}
Seiring dengan perkembangan teknologi dan tuntutan industri yang semakin kompleks, kebutuhan akan tenaga kerja yang memiliki skill dan pengetahuan praktis di bidang instrumentasi dan kontrol semakin meningkat. Oleh karena itu, Program Studi Sarjana Terapan {\prodi}, {\departemen}, {\fakultas}, {\universitas}, menyelenggarakan kegiatan Praktik Industri (PI) sebagai salah satu upaya untuk membekali mahasiswa dengan pengetahuan dan pengalaman praktis di dunia kerja. Mahasiswa dituntut untuk menggabungkan pengetahuan teoritis dari perkuliahan dengan kemampuan praktis. Hal ini dapat dicapai melalui praktikum di kelas dan PI di dalam atau luar lapangan selama masa studi.

Kegiatan Praktik Industri (PI) merupakan mata kuliah wajib sebagai salah satu syarat kelulusan bagi mahasiswa Program Studi Sarjana Terapan {\prodi}, {\departemen}, {\fakultas}, {\universitas}. Kegiatan ini bertujuan untuk mengimplementasikan ilmu teoritis yang telah diperoleh di kelas dalam dunia kerja nyata.

Pada era globalisasi, penting bagi perguruan tinggi untuk terus meningkatkan kualitas sumber daya manusia. Dengan mengacu pada konsep Tri Dharma Perguruan Tinggi sebagai pedoman utama, upaya terus dilakukan untuk meningkatkan kualitas mahasiswa baik dalam aspek softskill maupun kemampuan akademik. Pendidikan, penelitian, dan pengabdian masyarakat menjadi fokus utama dalam Tri Dharma Perguruan Tinggi. Salah satu cara untuk melaksanakan pendidikan dan penelitian adalah dengan melibatkan mahasiswa dalam berbagai institusi yang terkait dengan penerapan ilmu dalam praktiknya. Dengan demikian, diharapkan mahasiswa dapat mengembangkan kemampuan menggunakan metode ilmiah, memahami prosedur kerja, mengikuti aturan, serta beradaptasi dalam lingkungan kerja yang sesungguhnya.

Program Studi Sarjana Terapan {\prodi} di {\departemen}, {\fakultas}, {\universitas} mewajibkan kegiatan Praktik Industri sebagai mata kuliah wajib untuk setiap mahasiswa. Agar mahasiswa dapat berpengalaman langsung di lingkungan industri yang menerapkan ilmu dan teknologi kelistrikan dalam berbagai bidang seperti kontrol proses, instrumentasi industri, sistem sensor, sistem kendali, sistem komputer, dan robotika. Melalui Praktik Industri, diharapkan mahasiswa dapat memperluas pengetahuan dan pemahaman mereka tentang disiplin ilmu elektronika dan instrumentasi, serta mendapatkan wawasan mengenai konsep-konsep yang bersifat non-akademis dan non-teknis dalam dunia kerja. Selain itu, Praktik Industri juga memberikan kesempatan bagi mahasiswa untuk memberikan kontribusi pengetahuan kepada instansi terkait secara konsisten.

Dalam rangka memenuhi kewajiban magang sebagai wujud pendidikan praktisi dan penelitian, {\perusahaan} diharapkan dapat menjadi mitra untuk penerapan dasar instrumentasi dan kendali di industri yang telah didapatkan mahasiswa di perkuliahan. Hal ini sejalan dengan tujuan PI untuk membekali mahasiswa dengan pengetahuan dan pengalaman praktis di dunia kerja.

\section{Tujuan Umum}

Terdapat beberapa tujuan umum yang ingin dicapai oleh mahasiswa dalam pelaksanaan praktik industri sebagai berikut:

\begin{enumerate}
    \item Mahasiswa mendapatkan pengalaman praktis dengan menerapkan materi perkuliahan yang dipelajari dalam Program Studi Sarjana Terapan {\prodi}.
    \item Mahasiswa Mahasiswa dapat memperoleh pengetahuan dan pengalaman mengenai pola kerja dan perilaku kerja profesional nyata di dunia industri.
    \item Memperkenalkan mahasiswa pada dunia industri untuk membangun hubungan yang baik antara industri dan pendidikan.
    \item Memberikan pengalaman dan pengetahuan tambahan di bidang sistem instrumentasi dan kontrol di luar lingkungan perkuliahan.
    \item Memungkinkan mahasiswa untuk memahami dan mempelajari sistem instrumentasi dan kontrol dalam konteks praktis di luar lingkungan perkuliahan.
    \item Menerapkan pengetahuan yang diperoleh selama perkuliahan untuk menangani masalah troubleshooting dan menganalisis kesalahan yang mungkin terjadi.
\end{enumerate}

\section{Manfaat Praktik Industri}

\subsection{Bagi Perguruan Tinggi}

\begin{enumerate}
    \item Terjalinnya hubungan baik antara Program Studi Sarjana Terapan {\prodi} {\universitas} dengan {\perusahaan} sehingga memungkinkan kerja sama kedua belah pihak.
    \item Mendapat \textit{feedback} untuk meningkatkan kualitas pendidikan sehingga selalu sesuai dengan perkembangan dunia industri. 
\end{enumerate}

\subsection{Bagi Perusahaan}

\begin{enumerate}
    \item Lembaga pendidikan mendapatkan masukan baru melalui mahasiswa yang sedang atau sudah menjalani praktik industri.
    \item Dapat membangun hubungan yang baik dengan {\fakultas} {\universitas}, sehingga universitas tersebut mengakui {\perusahaan} sebagai penyedia tenaga kerja potensial dan masyarakat.
\end{enumerate}

\subsection{Bagi Mahasiswa}

\begin{enumerate}
    \item Mahasiswa mendapatkan pemahaman mendalam tentang kondisi aktual sebuah perusahaan atau industri, termasuk aspek manajemen yang diterapkan, kondisi fisik fasilitas, teknologi yang digunakan, kinerja karyawan, dan proses produksi yang terjadi di industri tersebut.
    \item Mahasiswa dapat mengetahui dan memahami sistem kerja perusahaan dan turut serta dalam proses.
    \item Mahasiswa mendapatkan pengalaman yang berguna untuk meningkatkan keterampilan teknis yang terkait dengan program studinya.
    \item Mahasiswa memiliki pemahaman tentang perkembangan ilmu dan teknologi yang sesuai dengan kemajuan industri, serta kemampuan untuk mengikuti perkembangan tersebut.
    \item Mahasiswa memiliki kesempatan untuk menjalin hubungan yang baik dengan industri sehingga memungkinkan mereka untuk bekerja di industri di mana mereka melaksanakan praktik industri.
\end{enumerate}
